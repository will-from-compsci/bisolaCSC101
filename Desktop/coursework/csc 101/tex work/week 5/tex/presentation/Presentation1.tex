\documentclass{article}
\begin{document}
	\title{A Historical View of Programming Languages}
	\author{Bisola Akinkugbe}
	\maketitle
	\section*{Python}
	\subsection*{What is Python?}
	\paragraph{}Python is an interpreted high-level general purpose code.
	\paragraph{} It is an object oriented language, which enables programmers to write concise, logical code for projects of any size.
	\paragraph{} \textit{Source: Wikipedia, “Python (programming language)”}
	\subsection*{History}
	\paragraph{}The name “Python” came from a TV show called “Monty Python’s Flying Circus”, which the creator of the language greatly enjoyed.
	\paragraph{}\textit{Source: JavaTPoint, “Python History and Versions”} \paragraph{}Python was invented as a successor to the ABC programming language, which was invented in 1987 by Geurts, Meertens and Pemberton.
	Python itself was invented by Dutch scientist Guido van Rossum at CWI 
	\paragraph{}Its implementation began in 1989 and the first stable build was released in 1991.
	Since then, multiple builds have been released, the latest of them being on October 4, 2021.
	\subsection*{Founder}
	\paragraph{}Python was invented by renowned scientist Guido van Rossum, born on January 31, 1956.
	He originates from the Netherlands and has since moved to California with his family.
	\paragraph{}He developed Python while working at the Centrum Wiskunde and Informatica (abbr. CWI) of the Netherlands Organization for Scientific Research.
	\textit{Source: Wikipedia, “Guido van Rossum”}
	\subsection*{Applications that can be made with Python}
	\begin{itemize}
		\item Games, such as: Doki Doki Literature Club, Toontown Online, Battlefield 2, etc.
		\item Hardware, such as: Robots, Machines, Motion sensors, etc.
		\item Applications, such as: Instagram, Spotify, Reddit, Google, etc.
	\end{itemize}
	\subsection*{IDEs for Python}
	\begin{itemize}
		\item Pycharm
		\item PyDev
		\item Spyder
		\item Scite
		\item Sublime Text, etc.
	\end{itemize}
	\subsection*{Why I Chose Python}
	\paragraph{}I chose Python because I have an extensive history with it.
	During the pandemic, I used it to create small scrolling games and edit some of the games I already had.
	It is my go-to language when I think of coding.
	\subsection*{Related Languages}
	\begin{itemize}
		\item ABC
		\item Haskell
		\item Modula-3
		\item Smalltalk
		\item ALGOL 68, etc.
	\end{itemize}


	\section*{BASIC}
	\subsection*{What is BASIC?}
	\paragraph{}BASIC stands for Beginners’ All-Purpose Symbolic Instruction Code.
	It is a general-purpose, high-level programming language.
	\paragraph{}\textit{Source: Wikipedia, “BASIC”}
	\paragraph{}BASIC encompasses a family of languages, including: Atari BASIC, Microsoft BASIC, QuickBASIC, Visual BASIC, etc.
	\subsection*{History}
	\paragraph{}BASIC was designed by Thomas Kurtz and John Kemeny at Dartmouth College, New Hampshire in May 1964.
	\paragraph{}It drew inspiration from FORTRAN II, which had many confusing commands, and simplified them to an easily understandable degree.
	\paragraph{}It was used mainly as an avenue for undergraduate students to write their own computer programs.
	\textit{Source: techterms.com, thoughtco.com}
	\subsection*{Founders}
	\subsubsection*{Thomas Kurtz}
	\paragraph{}He was born in Illinois, USA in February 22, 1928.
	He is a Ph.D holder and specialized in the area of mathematics and computer science.
	He was a professor at Dartmouth College, before his retirement from the tech field all together.
	He co-developed BASIC in 1964.
	\paragraph{}\textit{Source: Wikipedia, “Thomas E. Kurtz”}
	\subsubsection*{John Kemeny}
	\paragraph{}He was born in Hungary in May 31, 1926.
	Due to being Jewish, he moved to the USA with his family in 1940, following the rise of antisemitic sentiments.
	He is a computer scientist and mathematician and was the 13th President of Dartmouth College.
	He is said to have pioneered the use of computers in education.
	\paragraph{}\textit{Source: Wikipedia, “John G. Kemeny”}
	\paragraph{}He co-developed BASIC in 1964.
	He died in December 1992.
	\subsection*{Applications that can be made with BASIC}
	\begin{itemize}
		\item Games, such as: Gorillas, Nibbles, etc
		\item Applications, such as: Calculators, Clients, etc.
	\end{itemize}
	\subsection*{IDEs for BASIC}
	\begin{itemize}
		\item Qbasic
		\item QB64
		\item BASIC256
		\item Visual Basic. NET, etc.
	\end{itemize}
	\subsection*{Why I Chose BASIC}
	\paragraph{}I chose BASIC because it was my first introduction to the world of prgramming.
	We started to learn about it in JS1, and it gave me my first taste of Software Engineering.
	\subsection*{Related Languages}
	\begin{itemize}
	\item FORTRAN II
	\item JOSS
	\item ALGOL 60
	\end{itemize}
	\textit{Source: Wikipedia, “BASIC”}
	
	
	\section*{Java}
	\subsection*{What is Java?}
	\paragraph{}Java is a general-purpose, high-level, class-based, object-oriented language.
	\paragraph{}\textit{Source: guru99.com}
	\paragraph{}It is used to develop applications and scripts in the .json format.
	It is a server-side language for a lot of back-end developments.
	\paragraph{}\textit{Source: blogs.oracle.com}
	\subsection*{History}
	\paragraph{}Java was designed by James Gosling in 1995 and was published by Sun Microsystems, which is now a part of Oracle.
	They began working on the language in 1991, though it was published four years later.
	\subsection*{Founder}
	\paragraph{}James Gosling was born in Alberta, Canada, in May 19, 1955.
	He is a computer scientist by trade.
	He was elected as a member of the National Academy of Engineering in 2004 for his creation of Java
	\paragraph{}\textit{Source: Wikipedia, “James Gosling”}
	\subsection*{Applications that can be made with Java}
	\begin{itemize}
	\item Games such as: Minecraft, God of War, etc.
	\item Applications, such as: Web apps, LinkedIn, Uber, etc.
	\item Software, such as: Operating Systems, GUIs etc.
	\end{itemize}
	\subsection*{IDEs for Java}
	\begin{itemize}
	\item BlueJ
	\item Jdeveloper
	\item Eclipse
	\item NetBeans
	\item Microsoft Visual Studio, etc.
	\end{itemize}
	\subsection*{Why I Chose Java}
	\paragraph{}I chose Java because one of my favorite games, Minecraft, was created with Java.
	Also, I hope to someday learn Java so I can make my own modifications to the game.
	\subsection*{Related Languages}
	\begin{itemize}
		\item Lisp
		\item Smalltalk
		\item C++, etc.
	\end{itemize}

\section*{HTML}
\subsection*{What is HTML?}
\paragraph{}HTML stands for Hypertext Markup Language.
\paragraph{}It is the computer language that is used to make most webpages and online applications.
It is used to create and edit the apperance of the text, images, fonts, colors and hyperlinks we see on webpages.
\subsection*{History}
\paragraph{}HTML was created in 1991 by Tim Berners-Lee, following his creation of the World Wide Web.
It was created with the aim of sharing information on the Internet.
It was then published in 1995 as HTML 2.0
The latest version is HTML 5, which was released in 2012.
\subsection*{Founder}
\paragraph{}Tim Berners-Lee, aka TimBL, was born in London on June 8, 1995.
\paragraph{}He is a computer scientist and the Father of the World Wide Web.
He has won multiple awards for his pioneering work on the Internet, as without it, the Web as we know it today would be non-existent.
\paragraph{}He is currently a professor at the Massachusetts Institute of Technology (abbr. MIT) in the USA.
\subsection*{Applications that can be made with HTML}
\begin{itemize}
	\item Webpages
	\item Client-side storage
	\item Web documents
	\item Internet applications, etc.
\end{itemize}
\subsection*{IDEs for HTML}
\begin{itemize}
	\item Notepad
	\item Albloom Prosite
	\item Flazio
	\item IM Creator
\end{itemize}
\textit{Source: blog.capterra.com}
\subsection*{Why I Chose HTML}
\paragraph{}I chose HTML because I have previous experience working with the language and I would like to eventually master it.
I aspire to create my own webpages using HTML.
\subsection*{Related Languages}
\begin{itemize}
	\item XML
	\item XHTML
	\item JSX
	\item CSS, etc.
\end{itemize}

\section*{C++}
\subsection*{What is C++?}
\paragraph{}C++, aka CXX, is a general-purpose, object-oriented language.
It is an extension of the “C” language.
\paragraph{}It is a compiled language, and is provided as a compiler by vendors such as Oracle, Free Software Foundation, etc
\paragraph{}\textit{Source: Wikipedia, “C++”}
\subsection*{History}
\paragraph{}It was developed by Bjarne Stroustrup in 1982, at Bell Laboratories. It was meant to be a successor to “C”, then known as “C with Classes”.
He added new features to the language, including: constants, allocation, single-line comments, etc.
\paragraph{}\textit{Source: Wikipedia, “C++”}
\paragraph{}It was first released commercially in 1985 as Cfront.
Since then, multiple versions of the language have been released, the latest being C++20 in December 2020.
\subsection*{Founder}
\paragraph{}Bjanre Stroustrup was born in Aarhus, Denmark on December 30, 1950.
He graduated from university with a Master’s in computer science and mathematics. He later received a PhD in computer science.
He has won multiple awards for his work on C++.
\paragraph{}\textit{Source: Wikipedia, “Bjarne Stroustrup”}
\subsection*{Applications that can be made with C++}
\begin{itemize}
	\item Web browsers, such as: Mozilla Firefox, Google Chrome, etc.
	\item Games, such as: Hollow Knight, Doom 3, etc.
	\item Operating systems, such as: Windows 95, 98, XP, etc.
\end{itemize}
\subsection*{IDEs for C++}
\begin{itemize}
	\item Visual Studio
	\item Eclipse CDT
	\item NetBeans
	\item Xcode, etc.
\end{itemize}
\subsection*{Why I Chose C++}
\paragraph{}I chose C++ because I am interested in learning the C family of languages, especially C++.
This is because one of my favorite games, Hollow Knight, was made with a C language.
\subsection*{Related Languages}
\begin{itemize}
	\item BCPL
	\item C
	\item C sharp
	\item Ratfor, etc.
\end{itemize}

\end{document}